\begin{figure}[!htbp]
\centering
\resizebox{\linewidth}{!}{%
\begin{tikzpicture}[thick]
  \begin{umlseqdiag}
    \umlactor[scale=0.6,no ddots,fill=white!0]{Alice}
    \umlobject[no ddots,fill=white!0,x=7]{ERC820 Registry}
    \umlobject[class=ERC777TokensRecipient,fill=white!0,x=13]{Carlos}
    \begin{umlcall}[dt=6,draw=BrickRed,fill=BrickRed!20,op={Contract Creation}]{Alice}{Carlos}
    \end{umlcall}
    \begin{umlcall}[dt=8,name=set,draw=BrickRed,fill=BrickRed!20,op={\shortstack[l]{setInterfaceImplementer(\\\qquad 0x0, ERC777TokensRecipient, Carlos)}}]{Alice}{ERC820 Registry}
      \begin{umlcallself}[draw=MidnightBlue,fill=MidnightBlue!20,op={getManager(Alice)},return=Alice]{ERC820 Registry}
      \end{umlcallself}
      \begin{umlcall}[dt=8,padding=3,draw=MidnightBlue,fill=MidnightBlue!20,return={ERC820\_ACCEPT\_MAGIC},op={\shortstack[l]{canImplementInterfaceForAddress(\\\qquad Alice, ERC777TokensRecipient)}}]{ERC820 Registry}{Carlos}
      \end{umlcall}
    \end{umlcall}
    \umlnote[x=3,y=-4.5,width=4cm,fill=solarizedbeige]{st-set}{Passing the zero address (0x0) as first parameter is a shortcut to consider the address from which the call originates, in this case Alice.}
  \end{umlseqdiag}
  \filldraw[draw=BrickRed,fill=BrickRed!20] (0,-7) circle (.2);
  \node at (1.2,-7) {Transaction};
  \filldraw[draw=MidnightBlue,fill=MidnightBlue!20] (3,-7) circle (.2);
  \node at (3.6,-7) {Call};
\end{tikzpicture}
}
\caption{Example of a regular address, Alice, deploying a contract Carlos and then setting Carlos as her implementation of \texttt{ERC777TokensRecipient}.}
\label{fig:erc820ProxyRegister}
\end{figure}
